
\documentclass[12pt]{amsart}
\usepackage{geometry} % see geometry.pdf on how to lay out the page. There's lots.
\geometry{a4paper} % or letter or a5paper or ... etc
\usepackage[T1]{fontenc}
\usepackage[latin9]{inputenc}
\usepackage{amsmath}
\usepackage{amsaddr}
\usepackage{hyperref}
\usepackage{dirtytalk}
\usepackage{float}
\usepackage{listings}
\usepackage{color}
\usepackage{tikz}
 
\definecolor{codegreen}{rgb}{0,0.6,0}
\definecolor{codegray}{rgb}{0.5,0.5,0.5}
\definecolor{stringcolor}{rgb}{0.7,0.23,0.36}
\definecolor{backcolour}{rgb}{0.95,0.95,0.92}
\definecolor{keycolor}{rgb}{0.007,0.01,1.0}
\definecolor{itemcolor}{rgb}{0.01,0.0,0.49}
 
\lstdefinestyle{mystyle}{
    %backgroundcolor=\color{backcolour},   
    commentstyle=\color{codegreen},
    keywordstyle=\color{keycolor},
    numberstyle=\tiny\color{codegray},
    stringstyle=\color{stringcolor},
    basicstyle=\footnotesize,
    breakatwhitespace=false,         
    breaklines=true,                 
    captionpos=b,                    
    keepspaces=true,                 
    numbers=left,                    
    numbersep=5pt,                  
    showspaces=false,                
    showstringspaces=false,
    showtabs=false,                  
    tabsize=2
}
{
    "git.enableSmartCommit": true,
    "git.autofetch": true
}
 
\lstset{style=mystyle}

\lstdefinelanguage{Swift}{
  keywords={associatedtype, class, deinit, enum, extension, func, import, init, inout, internal, let, operator, private, protocol, public, static, struct, subscript, typealias, var, break, case, continue, default, defer, do, else, fallthrough, for, guard, if, in, repeat, return, switch, where, while, as, catch, dynamicType, false, is, nil, rethrows, super, self, Self, throw, throws, true, try, associativity, convenience, dynamic, didSet, final, get, infix, indirect, lazy, left, mutating, none, nonmutating, optional, override, postfix, precedence, prefix, Protocol, required, right, set, Type, unowned, weak, willSet},
  ndkeywords={class, export, boolean, throw, implements, import, this},
  sensitive=false,
  comment=[l]{//},
  morecomment=[s]{/*}{*/},
  morestring=[b]',
  morestring=[b]"
}

\lstset{emph={Int,count,abs,repeating,Array}, emphstyle=\color{itemcolor}}


\title{Week 03}

\date{\today}

\lstset{style=mystyle}

%%% BEGIN DOCUMENT
\begin{document}
\maketitle


Name: Joshua Ludwig
\\

Collaborators (if any): 

\section{Tasks}

\subsection{Sequence Rotations}

Here you can see a Sequence AVL Tree P. Perform the operation P.delete\_at(8) and draw the resulting tree after each rotation performed during the delete\_at operation. 

\begin{tikzpicture}[
    level 1/.style ={sibling distance=12em},
    level 2/.style ={sibling distance=6em},
    level 3/.style ={sibling distance=3em},
    every node/.style = {shape=circle,
    draw, align=center}]

    \node {6}
      child {
        node  {4}
            child {node {11}
            child {node {3}
            child {node {2}}
            child [missing]}
        child {node {1}}}
        child {node {5}
            child {node {9}}
            child [missing]}
      }
      child {
        node {12}
        child {node  {7}}
      child {node  {10}
        child {node  {8}} 
        child [missing]
        }
    };

    \end{tikzpicture}

\section{Exercises/Problems}

\subsection{Wild One} 
Emelyne has a wild 2 year old and she hears sounds of crashing and breaking and destruction coming from his playroom downstairs. She needs to decide whether to confront him (and be subject to his wrath and screaming) or to let him keep destroying things (which he seems to love to do these days). She decides to consult an elite group of n stuffed animals scattered around the room she's in currently. Emelyne surveys the animals and compiles a list of n polls, where each poll is a tuple matching a
different stuffed animals name with their integer opinion on the topic. Opinion + c means they are for
confronting her 2 year old, while opinion - c means they are against confronting her son. Emelyne wants to generate a list containing the names of the log n animals having the strongest opinions (breaking ties arbitrarily), so she can meet with them to discuss. For this problem, assume that the record containing the polls is read-only access controlled (the material in classified), so any computation must be written to alternative memory. 
\\

(a) Describe a O(n) time algorithm to generate Emelyne's list. 
\\

To generate Emelyne's list, we need to select the n animals with the strongest opinions. We can do this using the following algorithm:
To accomplish the task, you need to go through the list of polls and create a minimum heap that has a size of log n.
For each poll, compare the absolute value of the opinion with the smallest absolute opinion in the heap.
If the current poll has a larger absolute value, replace the smallest absolute value in the heap with the current poll.
After going through all the polls, the heap will contain the log n animals with the strongest opinions.
This algorithm iterates through all polls and maintains a log n-sized min-heap, resulting in a time complexity of O(n).

(b) Now suppose Emelyne's computer is only allowed to write to at most O(log n) space.
Describe an O(n log log n)-time algorithm to generate the list.

To achieve O(log n) space complexity while maintaining efficient time complexity, the problem can be broken into smaller pieces and solved separately.
Split the list of polls into two halves.
Find the animals with the strongest opinions in each half by recursively searching through the log (n/2).
Merge the results of both halves to obtain the log n animals with the strongest opinions.
This method achieves O(log n) space complexity by only storing recursive call results.

This algorithm has a time complexity of O(n log log n) due to its recursive nature. The space complexity is O(log n) because only the results of recursive calls are stored.

\end{document}
